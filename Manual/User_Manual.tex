\documentclass[11pt]{report}

\usepackage{url}
\usepackage{graphicx}
\usepackage{xcolor}
\usepackage[cc]{titlepic}
\usepackage{booktabs}
\usepackage[colorlinks = True,
linkcolor = blue]{hyperref}

\title{Eel Tracker\\ \Large User Manual for version 1.0}
\author{James Campbell}
\titlepic{\vspace{5cm}
\includegraphics[scale=0.2,clip = true,trim = 0 0 0 0]{zebrafish.jpeg}}

\begin{document}

\maketitle

\tableofcontents

\newcommand{\screenshot}[2]{
\begin{center}
\includegraphics[width= #2 ]{#1}
\end{center}
}

\chapter{Keyboard Shortcuts for Manual Corrections}\label{keyboard}

\begin{table}[h!]
  \begin{center}
    \begin{tabular}{cl}
      \toprule
      Z, X & Back, forward tracked frame \\
      A & Redefine head and tail point (midline will be automatically detected again) \\
      S & Manually define midline points (Press enter to set, in between points will be interpolated) \\
      \bottomrule
    \end{tabular}
  \end{center}
\end{table}

\chapter{User Guide}

\section{Quick Start}

The following program has three parts:

\begin{enumerate}
  \item Video Analysis
  \item Video Post-Processing
  \item Tail beat analysis
\end{enumerate}

Here we'll provide a walkthrough for all three sections which will allow you to analyze tail beat frequencies from videos in a semi-automated process.  The first two parts are done in matlab, resulting in a \texttt{csv} file holding the coordinates of the eel's midline relative to the swimming vector.  The final tail beat analysis will be done in R.

\subsection{MATLAB Compiler Runtime}

The easiest way to deploy the MATLAB parts of the analysis is by using MATLAB Compiler Runtime (MCR).
This allows you to run compiled MATLAB scripts without holding a paid license. You can download MCR \url{here}{https://www.mathworks.com/products/compiler/mcr.html}.
This binary was compiled with MATLAB 2013a (64-bit, version 8.1, windows), so you will need to download that same version of MCR.
Note that you must download the same archetecture type that the binary was compiled on, so even if you are using a 64-bit computer, you still need to download the 32-bit version of the program to run this.
Also note that the MCR installation is invisible.
Once complete, there are no items added to your start menu and the MCR will automatically open when you double click the compiled binary.

\subsection{Get R packages}
For the tail-beat analysis, we've saved the code you need in two packages hosted on github under the user \texttt{RTbecard}.
These packages are not on the CRAN, but you can find instructions for installation of the page of each repository.

\begin{enumerate}
  \item ezPSD (\url{https://github.com/RTbecard/LUAnimBehav}) \\
  This will be used for the frequency analysis
  \item LUAnimBehav (\url{LUAnimBehav}) \\
  This is a general purpose package for animal behavior where we've stored the functions for automating the tail beat analysis.
\end{enumerate}

\subsection{Get MATLAB code/binary}
You can get the code for the MATLAB programe here: \url{https://github.com/RTbecard/EelTracker}.
If you want to use a full version of matlab, you start the program by running the script \texttt{GUI.m}.
If you are interested in the MCR version, you only need to download the exe file located in the folder \texttt{CompiledBinaries}.

\subsection{Video Analysis}

Here we'll walk through the video analysis.

\begin{enumerate}
  \item Start by starting the MATLAB program (double-click the exe if you are using MCR).

  \screenshot{01.PNG}{250pt}

  \item When open you will see a set of options for analysis.
  While most are self-explanatory, we will quickly explain some of the options here:
  \begin{itemize}
    \item Frame Interval: You can use this to process only every nth frame of the video.
    \item Reference Offset: This is used for automatically defining the reference images for the background subtration which is used for detection movement in the video.
    To detenmine what value this should be, watch your video and move to the starting frame of your analysis.
    Now, draw a box around the eel's location.
    Next play the video and time how long it takes for the eel to completely leave this bounding box you drew.
    Setting this value too low will result in detection errors, as the eel tracker uses frames n seconds before and after the current frame to create a reference image for motion detection.
    \item Midline Points: The number of points to automatically draw along the midline.
      Note these points are equally spaced along the major axis of the eel.
    \item Ignore last _ midline points:  If you are analysing a fish with a large soft tail, it can be useful to ignore the motion detected by the tail, as this will extend beyond the midline and cause problems for the spline estimation.
    Points will still be drawn for the ignored points, but they will be ignored when calculating a fitted spline for the midline.
    \item Loess Smooth Intensity: This value determines the intensity of smoothing.
    Smoothing helps avoid errors caused by outlier points.
    If set too high, important features of the midline may be missed.
    If set too low, the midline may not resemble a smooth curve, and outlier points will have a greater effect o the estimated midline.
    \item Detection Threshold: Higher values will make the motion analysis less sensitive.
    \item Min Object Size: All detected objects with a pixel size smaller than this will be ignored by the analysis.
  \end{itemize}

  \item Select a video and press analyse to begin the auto midline detection.
  You will see an output like the image below:

  \screenshot{02.PNG}{500pt}

  \item



\end{itemize}

\end{document}
